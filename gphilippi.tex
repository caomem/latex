%\documentclass[11pt]{scrartcl}
\documentclass[11pt]{article}
\usepackage[utf8]{inputenc}
\usepackage{cmap}
\usepackage[T1]{fontenc}
\usepackage{lmodern}
\usepackage[brazil]{babel}

%\usepackage{lrsmath}
\usepackage{url}
\usepackage{enumerate}
\usepackage{indentfirst}
\usepackage{acronym}

\usepackage{amsmath}
\usepackage{booktabs}
\usepackage{amssymb}
\usepackage{amsthm}
\usepackage{amscd}
\usepackage{amsfonts}
\usepackage{dsfont}
\usepackage{color}
\usepackage{multicol}
\usepackage{hyperref}

\usepackage[table]{xcolor}

\usepackage{graphicx}
\usepackage{tikz}
\usetikzlibrary{arrows,shapes,positioning,shadows,trees}


\tikzset{
  basic/.style  = {draw, text width=2cm, drop shadow, font=\sffamily, rectangle},
  root/.style = {basic, rounded corners=6pt, thin,align=center, fill=green!30,
                   text width=8em},
  level 2/.style = {basic, rounded corners=6pt, thin,align=center, fill=green!30,
                   text width=8em,sibling distance=10mm}
  level 3/.style = {basic, rounded corners=2pt, thin,align=center, fill=green!30,
                    innersep = 10mm,sibling distance=10mm}
}

\usepackage{graphicx}

\usepackage{3dplot}

\newtheorem{defi}{Definição}[section]
\newtheorem{teo}{Teorema}[section]
\newtheorem{cor}{Corolário}[section]
\newtheorem{conj}{Conjectura}[section]
\newtheorem*{propri}{Propriedades}
\newtheorem{pro}{Proposição}[section]
\newtheorem{lema}{Lema}[section]
\newtheorem{obs}{Observação}[section]
\newtheorem{ex}{Exemplo}[section]

\newtheorem*{DGP}{{\emph{Distance Geometry Problem (DGP)}}}

\newtheorem*{iMDGP}{{\emph{interval Molecular Distance Geometry Problem (iMDGP)}}}

\newtheorem*{MDGP}{{\emph{Molecular Distance Geometry Problem (MDGP)}}}

\newtheorem*{DMDGP}{{\emph{Discretizable Molecular Distance Geometry Problem (DMDGP)}}}

\newtheorem*{DDGP}{{\emph{Discretizable Distance Geometry Problem $k =3$ (DDGP$_3$)}}}

\newtheorem*{DVOP}{{\emph{Discretization Vertex Order Problem (DVOP)}}}

\usepackage[top=3cm,bottom=3cm,right=2.5cm,left=2.5cm]{geometry}

%\usepackage[backend=biber,style=numeric-comp]{biblatex}
%\usepackage{csquotes}
%\addbibresource{references_bibdesk_papers.bib}

% \usepackage{showlabels} 

\usepackage{geometry}
\geometry{a4paper,top=3cm,left=2.5cm,right=2.5cm}



%\def\Dew{\Delta w}
%\def\Dex{\Delta x}
%\def\Dey{\Delta y}
%\def\Dez{\Delta z}
%\def\Cset{\mathcal{C}}

\begin{document}

%\title{Projeto de Pesquisa}
%\author{Projeto de Atividades Acadêmicas  \\ \textsc{Felipe Delfini Caetano Fidalgo}\thanks{\tt{felipaomat@gmail.com}}}
%\date{Maio de 2015}  
%\thispagestyle{empty}
%
%\maketitle

\begin{titlepage}

\newcommand{\HRule}{\rule{\linewidth}{0.5mm}} % Defines a new command for the horizontal lines, change thickness here

\center % Center everything on the page
 
%----------------------------------------------------------------------------------------
%	HEADING SECTIONS
%----------------------------------------------------------------------------------------

\begin{center}
\includegraphics[scale=0.22]{logoufsc}
\end{center}

\vspace{1cm}

\textsc{\LARGE Universidade Federal de Santa Catarina}\\[0.5cm] % Name of your university/college
{\Large Centro Tecnológico, de Ciências Exatas e Educação \\ Departamento de Matemática}\\[1.5cm] % Major heading such as course name
\textsc{\Large TCC I\\Qualificação\vspace{1.5cm} \\ {Licenciatura em Matemática}}\\[2.0cm] % Minor heading such as course title

%\textsc{\LARGE Universidade Federal de Santa Catarina}\\[0.5cm] % Name of your university/college
%{\Large Centro de Blumenau \\ Departamento de Matemática}\\[1.5cm] % Major heading such as course name
%\textsc{\Large PIBIC \\ Programa Institucional de Bolsas de Iniciação Científica \vspace{1.5cm} \\ {\bf PROJETO DE PESQUISA}}\\[2.0cm] % Minor heading such as course title

%----------------------------------------------------------------------------------------
%	TITLE SECTION
%----------------------------------------------------------------------------------------

\HRule \\[0.4cm]
{ \LARGE \bfseries Introdução à Álgebra Geométrica} \\ [0.4cm] % Title of your document
\HRule \\[2.5cm]
 
%----------------------------------------------------------------------------------------
%	AUTHOR SECTION
%----------------------------------------------------------------------------------------

\begin{minipage}{1\textwidth}
	\begin{center} \large
		Guilherme Philippi (guilherme.philippi@hotmail.com),
		\vspace{0.5cm}
		\\
		\underline{\textsc{Orientador:}} \vspace{0.2cm}
		Felipe Delfini Caetano Fidalgo (felipe.fidalgo@ufsc.br).
	\end{center}
\end{minipage} \\[2cm]


{\large \today} % Date, change the \today to a set date if you want to be precise


\vfill % Fill the rest of the page with whitespace

\end{titlepage}

\tableofcontents


\section{Resumo}

O tema deste trabalho é 

Aplicações

\section{Introdução}


\section{Objetivos e Justificativa} \label{aims}


\subsection{Objetivo Principal}

O que foi exposto acima torna possível a contextualização do {\bf principal objetivo} deste projeto, que está em estudar todos os aspectos matemáticos do DGP e suas variações, bem como da verificação se será possível encontrar uma ordem para os vértices do sistema proposto, isto é, verificar a solução do DVOP aplicado à conformações de sistemas de Robôs Móveis e aos resultados computacionais.

\subsection{Objetivos Específicos}

Com o intuito de contemplar isto, ressaltamos alguns dos {\bf objetivos específicos}:

\begin{enumerate}[(1)]

\item Estudar formas viáveis (pelos vieses energético, de construção e precisão da medida) para obtenção das distâncias entre os elementos do sistema físico.

\item Estudar as possíveis distribuições dos Robôs Móveis em um sistema genérico, visando verificar quais conjuntos de dados possam ser garantidos como entradas para a construção do problema, tal como em.

\item Verificar a solução do DVOP aplicado ao problema proposto e estudar o ordenamento de vértices que se adeque aos objetivos do trabalho.

\item Caso consiga-se uma boa ordenação para os vértices, verificar a aplicação do algorítimo BP para a solução do DDGP proposto, se não, estudar outros algorítimos que possam solucionar o problema;

\item Estudar a complexidade computacional do algorítimo proposto aplicado as possíveis distribuições estudadas no item.

\item Simular computacionalmente o algoritmo para solução do problema com instâncias artificialmente geradas, dominando cada passo utilizado;

\item Aplicar o algoritmo estudado em estruturas de pequena escala, como instâncias reais do problema;

\item Ter contato matemático e computacional com problemas científicos;

\item Habituar-se a ler, citar e escrever documentos científicos;

\item Associar os objetos estudados como aplicações e generalizações de estudos teóricos e práticos nas disciplinas dos cursos de graduação.
\end{enumerate}


\subsection{Justificativa}

Este projeto é tanto viável para alunos de Matemática quanto das Engenharias, cursos presentes no Centro de Blumenau, visto que tem viés interdisciplinar.

Os pré-requisitos conceituais básicos tangem Física, Geometria Analítica, Álgebra Linear, Cálculo Diferencial e Integral, Automação e Programação.

Além disso, este tema está ganhando cada vez mais espaço na literatura, uma vez que esta é uma área em crescimento constante devido a grande quantidade de novas aplicações advindas do rápido desenvolvimento tecnológico das últimas décadas. Assim, este estudo dá aos alunos a possibilidade de participar de estudos de pesquisa a nível de Iniciação Científica que possam contribuir, de maneira até significativa, com trabalhos em andamento de pesquisa com relevante nível acadêmico.

Isto, além de trazer pra prática o significado de vários estudos realizados no curso, ainda permite que o aluno tenha contato com ideias e experiências das quais possa fazer uso em eventual pós-graduação, como início de uma carreira acadêmica na área. Também, o contato com a literatura científica brasileira e estrangeira.

Por fim, ainda existem poucos pesquisadores no Brasil trabalhando nesta área que é bem ampla, como já foi comentado até agora neste texto. Ou seja, este projeto ainda tem por prerrogativa contribuir para a formação de pessoal que deve dar prosseguimento aos estudos na área, aumentando a quantidade de especialistas em GD no futuro, tornando o Brasil como uma referência.


\section{Metodologia e Resultados Esperados}
 
Pretende-se focar, inicialmente, no estudo teórico, utilizando as referências \cite{savvides2001dynamic}, , \cite{carlile:MinimalOrder}, \cite{carlileGDandAplications} e \cite{carlile:DDGP}, além de artigos e livros citados como referências nestes itens da Literatura. 

Este trabalho deve-se dar em separado na primeira parte, com encontros periódicos com o orientador, seja em forma de reunião ou de exposição oral, para apurar os desenvolvimentos teóricos e sanar eventuais dúvidas que surjam. Paralelamente, deve-se pedir que o aluno escreva o que está aprendendo, utilizando LaTeX, a fim de aprender a escrever cientificamente sobre o que se estuda.

Espera-se conseguir aplicar o DVOP de forma satisfatória ao sistema proposto, de forma a obter uma boa ordenação aos Robôs Móveis, assim, conseguindo montar o DDGP e aplicar o algorítimo BP. Caso não se consiga fazê-lo, espera-se poder estudar e aplicar outros algorítimos presentes ou não na literatura para propor uma solução ao problema. 

Contudo, o essencial, espera-se desenvolver uma implementação do algoritmo a fim, estimando o desempenho computacional no tocante à resolução de um DGP - tanto artificial como real, visando estudar sua complexidade computacional. Por fim, pretende-se elaborar um documento final com tais conteúdos e com simulações computacionais para indicar a validade científica do mesmo.

Pretende-se participar de reuniões científicas a fim de trocar experiências e idéias com outros alunos e pesquisadores, bem como estabelecer uma rede de contatos e colaborações.

Além disso, deseja-se escrever e submeter um artigo a nível de iniciação Científica a algum periódico nacional.

\section{Planos de Trabalho e Cronogramas}

Segue abaixo os planos de trabalho e respectivos cronogramas, ajustados para um ano de vigência, prevendo inclusive as escritas dos relatórios
	
	\subsection{Plano de Trabalho}
	
	\begin{enumerate}[(1)]
		
	\item {\bf Levantamento bibliográfico sobre Sistemas de Automação em literatura específica, formas de se obter distâncias, além de estudo sobre estruturas e funções destes sistemas}
	
	\vspace{0.2cm}
	
	Nesta atividade, o bolsista deverá juntamente com o orientador fazer um apanhado de artigos e livros referentes ao tema que servirão de apoio teórico. ele durará metade do período porque frequentemente novas bibliografias deverão ser consultadas. Além disto, este é o tempo onde o bolsista terá a oportunidade de estudar a fundo e propor formas de se obter os valores de distâncias medidos entre os elementos dos sistemas.
	
	\item {\bf Levantamento bibliográfico sobre o DVOP, DDGP e estudo do problema}
	
	\vspace{0.2cm}
	
	Nesta atividade, o bolsista deverá juntamente com o orientador fazer um apanhado de artigos e livros referentes ao tema que servirão de apoio teórico. ele durará metade do período porque frequentemente novas bibliografias deverão ser consultadas. Já o estudo do tema será dividido em partes: 
	
	\begin{enumerate}[(i)]
		\item estudo da Teoria de Grafos;
		
		\item estudo dos modelos de DGP usando grafos;
		
		\item modelar um DGP que caracterize o sistema real estudado
		
		\item modelagem e estudo de solução do DVOP sobre o sistema real;
		
		\item estudo extenso sobre algorítimos existentes para solucionar o problema;
		
		\item estudar a factibilidade da solução do problema e, caso for possível, propor um algorítimo que o solucione.
		
	\end{enumerate}
	
	 Todas as contas devem ser abertas e teoremas demonstrados. Este item e o anterior estão ligados e durarão o mesmo tempo porque devem ser estudados em paralelo, já que são os temas principais e co-relacionados fortemente.
	
	\item {\bf Escrita do relatório parcial referente à primeira metade da vigência, contemplando os dois itens anteriores}
	
	Nesta atividade, que durará a primeira metade do projeto, o aluno deve escrever parcialmente o relatório periodicamente a cada avanço teórico realizado. Dever-se-á estar em paralelo com os itens anteriores.
	
	\item {\bf Estudo detalhado e minucioso (teórico e computacional) do Algoritmo Proposto}
	
	Nesta atividade, que durará o quarto e quinto bimestre, o bolsista deverá estudar em detalhes a estrutura algorítmica do algoritmo que fora proposto no item anterior. Este estudo será dividido em partes:
	
	\begin{enumerate}[(i)]
		\item estudo sobre a Teoria de Complexidade Computacional
		
		\item verificar a complexidade computacional do algorítimo adotado
		
		\item estudar formas de se alterar o sistema afim de se obter melhores soluções computacionais
	\end{enumerate}
	
	\item {\bf Simulações em conjunto com instâncias artificiais e reais}
	
	Nesta atividade, o bolsista deverá simular numericamente com instâncias artificiais e reais, usando o algoritmo implementado por ele. Durará toda a segunda metade do projeto, visto que a cada passo que se estuda deve-se realizar as implementações computacionais.
	
	\item {\bf Submissões a possíveis encontros científicos, o que em nossa área ocorre no primeiro semestre do ano}
	
	Nesta atividade, o bolsista deverá aprender a escrever propostas, resumos, posteres e, eventualmente, se comunicar oralmente sobre o objeto de seus estudo científico.
	
	\item {\bf Escrita do relatório final}
	
	Nesta atividade, o bolsista deverá finalizar o relatório parcial com a escrita do relatório final, aparando as arestas e fazendo as correções necessárias apontadas pelo orientador.
	\end{enumerate}
	
	
	\subsection{Cronograma}
	
\begin{table}[h]
\centering
	\begin{tabular}{| c || c | c | c | c | c | c |}
	\hline 
	Atividades & 1$^{\circ}$ bimestre & 2$^{\circ}$ bimestre & 3$^{\circ}$ bimestre & 4$^{\circ}$ bimestre  & 5$^{\circ}$ bimestre & 6$^{\circ}$ bimestre \\ \hline \hline
	(1) & \cellcolor{gray} & \cellcolor{gray} & \cellcolor{gray} &  & &\\ \hline
	(2) & \cellcolor{gray} & \cellcolor{gray} & \cellcolor{gray} &  & &\\ \hline
	(3) & \cellcolor{gray} & \cellcolor{gray} & \cellcolor{gray} & \cellcolor{gray} &\cellcolor{gray} & \\ \hline
	(4) &  & & & \cellcolor{gray} & \cellcolor{gray} & \\ \hline
	(5) &  & & & \cellcolor{gray} & \cellcolor{gray} & \cellcolor{gray} \\ \hline
	(6) &  & & & & & \cellcolor{gray} \\ \hline
	(7) &  & & &  & & \cellcolor{gray} \\
	\hline
	\end{tabular}
\end{table}
	

\section{Exequibilidade}

Devida a natureza das implementações propostas no projeto, o aluno realizará um trabalho de pesquisa em livros e artigos, de simulações em seus computadores pessoais, utilizando o software livre Octave (uma versão livre com recursos similares ao Matlab, ou outro, dependendo da preferência do estudante) e poderá implementar um ambiente real (utilizando microcontroladores como Arduíno ou similares, dependendo da preferência do bolsista) afim de aplicar a solução proposta (se existir). O ambiente de estudo deve ser escolhido pelo aluno. As reuniões e seminários serão realizados em alguma sala-de-aula na Sede Acadêmica do Centro.

\phantomsection
\addcontentsline{toc}{section}{Referências}

\bibliographystyle{unsrt}
\bibliography{references}

\end{document}