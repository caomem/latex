\documentclass[a4paper,12pt]{article}
\usepackage[a4paper,top=3cm,bottom=2cm,left=3cm,right=3cm,marginparwidth=1.75cm]{geometry}
\usepackage[brazil]{babel}
\usepackage[T1]{fontenc}
\usepackage[utf8]{inputenc}
\usepackage{amsmath}
\usepackage{MnSymbol}
\usepackage{wasysym}
\usepackage{hyperref}
\usepackage{color}
\definecolor{Blue}{rgb}{0,0,0.9}
\definecolor{Red}{rgb}{0.9,0,0}
\usepackage{esvect}
\usepackage{graphicx}
\usepackage{float}
\usepackage{indentfirst}
\usepackage{caption}
\usepackage{blkarray}
\newcommand\Mark[1]{\textsuperscript#1}
\usepackage{pgfplots}
\usepackage{amsfonts}
\usepackage[english, ruled, linesnumbered]{algorithm2e}
\usepackage{algorithmic}
\newtheorem{definicao}{Definição}[section]
\newtheorem{teorema}{Teorema}[section]
\newtheorem{lema}{Lema}[section]
\newtheorem{proposicao}{Proposição}[section]
\newtheorem{observacao}{Observação}[section]
\newtheorem{corolario}{Corolário}[section]

\title{Notas de Estudo sobre Otimização Contínua}
\author{Guilherme Philippi}
\begin{document}
\maketitle
\tableofcontents

% - O que é Modelagem Matemática de um problema (características);
% - O que é um problema de otimização e pra que ela serve;
% - O que é otimização discreta e otimização contínua;
% - Os tipos de programação de cada uma delas;


\section{Introdução}

A otimização é uma área da matemática que objetiva encontrar soluções ótimas para um problema. Entende-se como solução ótima aquela que minimizará ou maximizará uma função, chamada de \textit{função objetivo} do problema. Nem sempre essa solução ótima será fácil ou possível de encontrar, donde essa área também se restringe ao estudo de \textit{aproximações} para as soluções ótimas de um problema.

Definir a função objetivo de um problema de otimização é parte da \textit{modelagem matemática} do problema real. Para isso, busca-se descrever as características e comportamentos de uma situação do mundo real através de variáveis e funções que as restrinjam, abstraindo a situação real de tal forma que não se percam os comportamentos de interesse. Quanto mais variáveis e funções utiliza-se para descrever matematicamente os comportamentos do sistema real, mais fidedigna e complexa a sua representação se torna.

Um problema de otimização é representado como segue: 
$$\min_{}$$



\phantomsection
\addcontentsline{toc}{section}{Referências}

\bibliographystyle{unsrt}
\bibliography{references}

\end{document}
