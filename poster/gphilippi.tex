\documentclass[a4,11pt]{pssbmac}

\usepackage[brazil]{babel}      % para texto em Português
%\usepackage[english]{babel}    % para texto em Inglês

%\usepackage[latin1]{inputenc}   % para acentuação em Português OU
\usepackage[utf8]{inputenc}   % para acentuação em Português com o uso do Unicode, 
% mude a codificação desse padrão para utf-8

%%
%% POR FAVOR, NÃO FAÇA MUDANÇAS NESSE PADRÃO QUE ACARRETEM  EM
%% ALTERAÇÃO NA FORMATAÇÃO FINAL DO TEXTO
%%
\usepackage{graphics}
\usepackage{subfigure}
\usepackage{graphicx}
\usepackage{epsfig}
\usepackage[centertags]{amsmath}
\usepackage{graphicx,indentfirst,amsmath,amsfonts,amssymb,amsthm,newlfont}
\usepackage{longtable}
\usepackage{cite}
\usepackage[usenames,dvipsnames]{color}
\usepackage{booktabs}
\begin{document}
	
	%********************************************************
	\title{Um estudo teórico-computacional da aplicação da Geometria de Distâncias no problema de conformação proteica}
	
	\author{
		{\large Guilherme Philippi}\thanks{guilherme.philippi@hotmail.com}\\
		{\small UFSC, Blumenau, SC} \\
		{\large Felipe Fidalgo}\thanks{felipe.fidalgo@ufsc.br} \\
		{\small MAT/UFSC, Blumenau, SC} \\
	}
	\criartitulo
	
	\textbf{x}$\mathbf{x}_{i-3}$
	
	A Geometria de Distâncias originou-se dos esforços de Menger (1928), seguido por Blumenthal (1953), na caracterização de vários conceitos geométricos (como congruência e convexidade de conjuntos) em termos de distâncias \cite{carlileGDandAplications}. Desse estudo nasceu o \textit{Distance Geometry Problem} (DGP), conhecido como problema fundamental da Geometria de Distâncias. Trata-se de um problema inverso, onde, dado um grafo ponderado positivamente, não direcionado e simples $G=(V,E, d)$, e um inteiro $k>0$, deseja-se encontrar uma função $x:V\rightarrow\mathbb{R}^k$ (dita imersão de G em $\mathbb{R}^k$) tal que $\forall \{u,v\} \in E$, $||x(u) - x(v)$
	
	Nesta categoria os trabalhos podem ser submetidos em Português ou Inglês e serão apresentados na forma de paineis, dentro da sessão técnica onde está inscrito. Para que seja aceito e incluído na Programação do evento, é necessário que o {\bf autor-apresentador tenha pago a taxa de inscrição até a data definida na página do CNMAC}. Cada taxa de inscrição permite a apresentação de, no máximo, 2 (dois) trabalhos, em quaisquer categorias, respeitadas as restrições sobre o tipo e categoria das contribuições. Desse modo, cada participante só poderá submeter até 2 (dois) trabalhos.
	
	Os trabalhos aceitos e apresentados serão publicados no {\it Proceeding Series of the Brazilian Society of Computational and Applied Mathematics} \footnote{http://proceedings.sbmac.org.br/sbmac}. Por esta razão, ao submeter e apresentar um trabalho, fica o autor ciente que o mesmo será publicado pela SBMAC, sendo tacitamente cedidos os direitos autorais à Sociedade.
	
	Para equações, figuras e tabelas, orientamos que seja seguido o modelo disponível para a Categoria 2, de trabalhos completos.
	
	{\bf As referências devem estar em ordem alfabética pelo sobrenome do primeiro autor.} Cada referência é produzida através do comando  \verb!\bibitem{nome-da-ref}! e é citada no texto via o comando \verb!\cite{nome-da-ref}!. A bibliografia ({\it Referências}), que deve figurar no final do artigo, é então gerada da seguinte forma (exige dupla compilação):
	
	\begin{verbatim}
	\begin{thebibliography}{00}
	\bibitem{}
	\end{thebibibliography}
	\end{verbatim}
	%%
	
	Abaixo incluímos modelos de diversos tipos de publicações. Um padrão similar deve ser usado para os casos omitidos aqui.
	
	\begin{itemize}
		
		\item No caso de livros, deve-se seguir o padrão da referência \cite{Boldrini}, ou, para aqueles publicados dentro de uma série, \cite{Gomes}. Se capítulo de livro, após o título da publicação,  deve vir o título da série (quando aplicável), o número do capítulo e o volume, como na referência \cite{daSilva};
		\item Se artigo, após o título da publicação ({\it em itálico}) deve vir o volume e as páginas correspondentes, seguidos do ano, conforme as referências \cite{Diniz2}. Trabalhos aceitos, mas não publicados, devem ser citados conforme mostrado na referência \cite{Cuminato}. Não havendo DOI, coloque-se (to appear). Trabalhos publicados em anais de eventos devem seguir o padrão da referência em \cite{Santos};
		\item Dissertações, teses e similares devem seguir o padrão da referência \cite{Diniz1}.\cite{AlvesAACA_2015}
	\end{itemize}
	
	\section*{Agradecimentos (opcional)}
	Seção reservada aos agradecimentos dos autores, caso pertinente.
	
	\addcontentsline{toc}{section}{Referências}
	
	\bibliographystyle{unsrt}
	\bibliography{References/references}
\end{document}